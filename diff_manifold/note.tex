\documentclass[11pt,a4paper]{ctexbook}
%\usepackage{CJKutf8}
%\setCJKmainfont{AR PL UKai CN}
%\setCJKmainfont{AR PL UKai}
\usepackage[T1]{fontenc}
\usepackage{lmodern}
\usepackage{url}
\usepackage[svgnames]{xcolor}
%% Some shades
\definecolor{Dark}{gray}{0.2}
\definecolor{MedDark}{gray}{0.4}
\definecolor{Medium}{gray}{0.6}
\definecolor{Light}{gray}{0.8}
%\ifpdf
%\usepackage{pdfcolmk}
%\fi
%% check if using xelatex rather than pdflatex
%\ifxetex
\usepackage{fontspec}
%\fi
\usepackage{graphicx}
%%\usepackage{hyperref}
%% drawing package
\usepackage{tikz}
%% for dingbats
\usepackage{pifont}
\usepackage{epigraph}
\usepackage{caption}
\usepackage{graphicx, subfig,float}
\geometry{a4paper,left=4cm,right=4cm}
\usepackage{appendix}
\usepackage{amsmath}
\usepackage{amssymb}
\usepackage[colorlinks,linkcolor=red,anchorcolor=blue,citecolor=blue]{hyperref}
\usepackage{slashed}
%\usepackage{simplewick}
%\usepackage{tikz}
\usepackage{tcolorbox}
%\usepackage[T1]{fontenc}
\usepackage{cleveref}
\usepackage{pdfpages}

%symbols
\newcommand{\pard}[3]{
	\left(\frac{\partial #1}{\partial #2}\right)_{#3}
}
\newcommand{\pd}[2]{
	\frac{\partial #1}{\partial #2}
}
\def\etot{E_{\text{total}}}
\def\ntot{N_{\text{total}}}
\def\calz{\mathcal{Z}}
\def\meanvl#1{\langle #1 \rangle}

%colors
\def\blacktext#1{{\color{black}#1}}
\def\bluetext#1{{\color{blue}#1}}
\def\redtext#1{{\color{red}#1}}
\def\darkbluetext#1{{\color[rgb]{0,0.2,0.6}#1}}
\def\skybluetext#1{{\color[rgb]{0.2,0.7,1.}#1}}
\def\cyantext#1{{\color[rgb]{0.,0.5,0.5}#1}}
\def\greentext#1{{\color[rgb]{0,0.7,0.1}#1}}
\def\darkgray{\color[rgb]{0.2,0.2,0.2}}
\def\lightgray{\color[rgb]{0.6,0.6,0.6}}
\def\gray{\color[rgb]{0.4,0.4,0.4}}
\def\blue{\color{blue}}
\def\red{\color{red}}
\def\green{\color{green}}
\def\darkgreen{\color[rgb]{0,0.4,0.1}}
\def\darkblue{\color[rgb]{0,0.2,0.6}}
\def\skyblue{\color[rgb]{0.2,0.7,1.}}
%%control
\def\be{\begin{equation}}
\def\ee{\nonumber\end{equation}}
\def\beq{\begin{equation}}
\def\eeq{\end{equation}}
\def\bea{\begin{eqnarray}}
\def\eea{\end{eqnarray}}
\def\nbea{\begin{eqnarray*}}
\def\neea{\nonumber\end{eqnarray*}}
\def\bmat#1{\left(\begin{array}{#1}}
\def\emat{\end{array}\right)}
\def\bcase#1{\left\{\begin{array}{#1}}
\def\ecase{\end{array}\right.}
\def\bmini#1{\begin{minipage}{#1\textwidth}}
\def\emini{\end{minipage}}
\def\tbox#1{\begin{tcolorbox}#1\end{tcolorbox}}
\def\pfrac#1#2#3{\left(\frac{\partial #1}{\partial #2}\right)_{#3}}
%%symbols
\def\bropt{\,(\ \ \ )}
\def\spa{\,\,\,\,}
\def\sone{$\star$}
\def\stwo{$\star\star$}
\def\sthree{$\star\star\star$}
\def\sfour{$\star\star\star\star$}
\def\sfive{$\star\star\star\star\star$}
\def\rint{{\int_\leftrightarrow}}
\def\roint{{\oint_\leftrightarrow}}
\def\stdHf{{\textit{\r H}_f}}
\def\deltaH{{\Delta \textit{\r H}}}
\def\ii{{\dot{\imath}}}
\def\skipline{{\vskip0.1in}}
\def\skiplines{{\vskip0.2in}}
\def\lagr{{\mathcal{L}}}
\def\hamil{{\mathcal{H}}}
\def\vecv{{\mathbf{v}}}
\def\vecx{{\mathbf{x}}}
\def\vecy{{\mathbf{y}}}
\def\veck{{\mathbf{k}}}
\def\vecp{{\mathbf{p}}}
\def\vecn{{\mathbf{n}}}
\def\vecA{{\mathbf{A}}}
\def\vecP{{\mathbf{P}}}
\def\vecsigma{{\mathbf{\sigma}}}
\def\hatJn{{\hat{J_\vecn}}}
\def\hatJx{{\hat{J_x}}}
\def\hatJy{{\hat{J_y}}}
\def\hatJz{{\hat{J_z}}}
\def\hatj#1{\hat{J_{#1}}}
\def\hatphi{{\hat{\phi}}}
\def\hatq{{\hat{q}}}
\def\hatpi{{\hat{\pi}}}
\def\vel{\upsilon}
\def\Dint{{\mathcal{D}}}
\def\adag{{\hat{a}^\dagger}}
\def\bdag{{\hat{b}^\dagger}}
\def\cdag{{\hat{c}^\dagger}}
\def\ddag{{\hat{d}^\dagger}}
\def\hata{{\hat{a}}}
\def\hatb{{\hat{b}}}
\def\hatc{{\hat{c}}}
\def\hatd{{\hat{d}}}
\def\hatN{{\hat{N}}}
\def\hatH{{\hat{H}}}
\def\hatp{{\hat{p}}}
\def\Fup{{F^{\mu\nu}}}
\def\Fdown{{F_{\mu\nu}}}
\def\newl{\nonumber \\}
\def\vece{\mathrm{e}}
\def\calM{{\mathcal{M}}}
\def\calT{{\mathcal{T}}}
\def\calR{{\mathcal{R}}}
\def\barpsi{\bar{\psi}}
\def\baru{\bar{u}}
\def\barv{\bar{\upsilon}}
\def\qeq{\stackrel{?}{=}}
\def\torder#1{\mathcal{T}\left(#1\right)}
\def\rorder#1{\mathcal{R}\left(#1\right)}
\def\contr#1#2{\contraction{}{#1}{}{#2}#1#2}
\def\trof#1{\mathrm{Tr}\left(#1\right)}
\def\trace{\mathrm{Tr}}
\def\comm#1{\ \ \ \left(\mathrm{used}\ #1\right)}
\def\tcomm#1{\ \ \ (\text{#1})}
\def\slp{\slashed{p}}
\def\slk{\slashed{k}}
\def\calp{{\mathfrak{p}}}
\def\veccalp{\mathbf{\mathfrak{p}}}
\def\Tthree{T_{\tiny \textcircled{3}}}
\def\pthree{p_{\tiny \textcircled{3}}}
\def\dbar{{\,\mathchar'26\mkern-12mu d}}
\def\erf{\mathrm{erf}}
\def\const{\mathrm{constant}}
\def\pheat{\pfrac p{\ln T}V}
\def\vheat{\pfrac V{\ln T}p}
%%units
\def\fdeg{{^\circ \mathrm{F}}}
\def\cdeg{^\circ \mathrm{C}}
\def\atm{\,\mathrm{atm}}
\def\angstrom{\,\text{\AA}}
\def\SIL{\,\mathrm{L}}
\def\SIkm{\,\mathrm{km}}
\def\SIyr{\,\mathrm{yr}}
\def\SIGyr{\,\mathrm{Gyr}}
\def\SIV{\,\mathrm{V}}
\def\SImV{\,\mathrm{mV}}
\def\SIeV{\,\mathrm{eV}}
\def\SIkeV{\,\mathrm{keV}}
\def\SIMeV{\,\mathrm{MeV}}
\def\SIGeV{\,\mathrm{GeV}}
\def\SIcal{\,\mathrm{cal}}
\def\SIkcal{\,\mathrm{kcal}}
\def\SImol{\,\mathrm{mol}}
\def\SIN{\,\mathrm{N}}
\def\SIHz{\,\mathrm{Hz}}
\def\SIm{\,\mathrm{m}}
\def\SIcm{\,\mathrm{cm}}
\def\SIfm{\,\mathrm{fm}}
\def\SImm{\,\mathrm{mm}}
\def\SInm{\,\mathrm{nm}}
\def\SImum{\,\mathrm{\mu m}}
\def\SIJ{\,\mathrm{J}}
\def\SIW{\,\mathrm{W}}
\def\SIkJ{\,\mathrm{kJ}}
\def\SIs{\,\mathrm{s}}
\def\SIkg{\,\mathrm{kg}}
\def\SIg{\,\mathrm{g}}
\def\SIK{\,\mathrm{K}}
\def\SImmHg{\,\mathrm{mmHg}}
\def\SIPa{\,\mathrm{Pa}}


\usepackage{wrapfig}
\graphicspath{{figures/}}
\usepackage{enumerate}
\makeatletter
\@addtoreset{equation}{section}
\makeatother
\renewcommand{\theequation}{\arabic{section}.\arabic{equation}}



%\usepackage{fancyhdr}
%\cpic{<尺寸>}{<文件名>}}用于生成居中的图片。
\newcommand{\cpic}[2]{
\includegraphics[scale=#1]{#2}
}

%\cpicn{<尺寸>}{<文件名>}{<注释>}用于生成居中且带有注释的图片,其label为图片名。
\newcommand{\cpicnl}[3]
{
\begin{wrapfigure}{l}[1cm]{0pt}
\cpic{#1}{#2}
\caption{#3\label{#2}}
\end{wrapfigure}
}

\newcommand{\cpicn}[3]
{
\begin{figure}
\begin{center}
\cpic{#1}{#2}
\caption{#3\label{#2}}
\end{center}
\end{figure}
}

\usepackage{titlesec} %自定义多级标题格式的宏包
\titleformat{\section}[block]{\LARGE\bfseries}{\arabic{section}}{1em}{}[]
\titleformat{\subsection}[block]{\Large\itshape\mdseries}{\arabic{section}.\arabic{subsection}}{1em}{}[]
\titleformat{\subsubsection}[block]{\normalsize\bfseries}{\arabic{section}.\arabic{subsection}.\arabic{subsubsection}}{1em}{}[]
\titleformat{\paragraph}[block]{\small\bfseries}{[\arabic{paragraph}]}{1em}{}[]
\usepackage[amsmath,thmmarks,framed]{ntheorem}
\usepackage{framed}
\definecolor{gray}{rgb}{0.6,0.6,0.6}
\renewcommand*\FrameCommand{{\color{gray}\vrule width 5pt \hspace{10pt}}}
\newframedtheorem{assumption}{假设}
\newframedtheorem{comments}{评论}
\newframedtheorem{problem}{练习}
\newframedtheorem{think}{思考题}
\graphicspath{{figure/}}
\usepackage{enumerate}
%\usepackage{fancyhdr}
%\cpic{<尺寸>}{<文件名>}}用于生成居中的图片。
\newcommand{\cpic}[2]{
\begin{center}
\includegraphics[scale=#1]{#2}
\end{center}
}
%\cpicn{<尺寸>}{<文件名>}{<注释>}用于生成居中且带有注释的图片,其label为图片名。
\newcommand{\cpicn}[3]
{
\begin{figure}[H]
\cpic{#1}{#2}
\caption{\color{red}#3\label{#2}}
\end{figure}
}

\newtheorem{definition}{\hspace{2em} 定义}[section]
\newtheorem{practice}{\hspace{2em} 练习}[section]
\newtheorem{theorem}{\hspace{2em} 定理}[section]
\newtheorem{problem}{\hspace{2em} 习题}[chapter]
\title{微分几何笔记}
\author{徐昊霆}
\begin{document}
\begin{titlepage} % Suppresses headers and footers on the title page

	\raggedleft % Right align everything
	
	\vspace*{\baselineskip} % Whitespace at the top of the page
	
	%------------------------------------------------
	%	Author
	%------------------------------------------------
	
	{\Large 徐昊霆} % Author name
	
	\vspace*{0.167\textheight} % Whitespace before the title
	
	%------------------------------------------------
	%	Title and subtitle
	%------------------------------------------------
	
	\textbf{\LARGE A Note of}\\[\baselineskip] % First title line
	
	{\textcolor{red}{\Huge Differentiable Manifold}}\\[\baselineskip] % Main title line which draws the focus of the reader
	
	{\Large \textit{For Physicists}} % Subtitle
	
	\vfill % Whitespace between the titles and the publisher
	
	%------------------------------------------------
	%	Publisher
	%------------------------------------------------
	
	{\large SPA, SYSU} % Publisher and logo
	
	\vspace*{3\baselineskip} % Whitespace at the bottom of the page

\end{titlepage}

\tableofcontents
\newpage
\chapter{微分流形与张量场}
\section{微分流形}\index{manifold}
\begin{definition}[开覆盖]
  一集合$X$的开子集的集合$\{O_\alpha\}$叫做$A\subset X$的一个开覆盖(open cover),如果$A\subset \cup_\alpha O_\alpha$,也可以说$\{O_\alpha\}$覆盖了$A$。
\end{definition}
\begin{definition}[ $n$维微分流形]
  
  对于拓扑空间$(M,\mathcal{T})$,如果$M$有开覆盖$\{O_\alpha\}$,$M\subset \cup_\alpha O_\alpha$,满足
  
  (a) 存在同胚映射$\psi_\alpha: O_\alpha \rightarrow V_\alpha$($V_\alpha$是$\mathbb{R}^n$用通常拓扑衡量的开子集)
  
  (b) $O_\alpha \cap O_\beta \ne \varnothing$, 则复合映射$\psi_\beta \circ \psi_\alpha^{-1}$(如图~\ref{manifold})是光滑的。
  
  则称拓扑空间$(M,\mathcal{T})$为$n$维微分流形,简称$n$维流形。
  
\end{definition}
\cpicn{0.5}{manifold}{微分流形的定义}

简单来说,开覆盖是很多小开集的集合,这些小开集把整个集合铺满了。现在又在每个小开集上定义了一个坐标网,即上面定义的同胚映射,把定义了坐标网的拓扑空间叫做微分流形。一个形象的比喻如图所示,如果把集合比作鱼,那么开集就是鱼身上的鳞片,开覆盖就是鱼身上鳞片的集合,如果每两个包含同一点的鳞片,那么就说鱼的表皮是连续的。如果在每一个鳞片上画一系列坐标网(即把每一点映射到2维数组上),那么鱼的表面就是一个二维微分流形。%\footnote{这个比喻来源于我的一位很好的朋友,他是一个小恐龙爱好者。原来的比喻是恐龙上的鳞片。}
\cpicn{0.4}{fish}{鱼和鱼的鳞片}
\section{流形上的切矢量}
我们先来定义流形上的标量场。
\begin{definition}[流形上的标量场]
  流形$\mathcal{M}$上的标量场是映射$f:\mathcal{M}\rightarrow \mathbb{R}$,即将流形上的一个点$p$映射到一个实数$f(p)$。流形$\mathcal{M}$上所有标量场的集合记为$\mathcal{F}_{\mathcal{M}}$
\end{definition}
例如,温度、压强、质量密度、电荷密度、高斯曲率等等,都是从一个点到实数轴上的映射,因此都是标量场。

有了流形上的标量场,我们来定义流形$\mathcal{M}$上点$p$的切矢量。
\begin{definition}[流形上某一点的{\color{red} 切矢量}]
  流形$\mathcal{M}$上点$p$的切矢量是标量场到实数的映射。即$X_p:\mathcal{F}_{\mathcal{M}}\rightarrow \mathbb{R}$,并且满足
  \begin{itemize}
  \item $X_p$是线性映射,即
    \be
    X_p(\alpha f +\beta g) =\alpha X_p(f) + \beta X_p(g)
    \ee
    其中$\alpha,\beta \in \mathbb{R}$。
  \item 满足莱布尼茨律
    \be
    X_p(fg) = f(p)X_p(g)+g(p)X_p(f)
    \ee
  \end{itemize}
\end{definition}
\begin{practice}
  利用切矢量的定义证明
  \begin{enumerate}
  \item $X_p(1) = 0$
  \item $X_p(c) =0$
  \item $X_p(\alpha f) = \alpha X_p (f)$
  \end{enumerate}
\end{practice}
\section{切空间--流形上的切矢量形成线性空间}
上面我们定义了切矢量是流形上标量场到实数轴的线性、满足莱布尼茨律的映射。在切矢量定义的基础上定义切矢量的加法和数乘,可以定义切空间\footnote{顾名思义,切矢量所形成的空间},记为$T_p\mathcal{M}$,意为流形上$p$点的切空间。
\begin{definition}[切矢量的线性运算]
  定义切矢量的加法为
  \be
  (X_p+Y_p)(f) = X_p(f) + Y_p(f)
  \ee
  存在一个特殊的$0_p\in T_p\mathcal{M}$,使得对于任意$f\in\mathcal{F}_{\mathcal{M}}$有$0_p(f) =0$,切矢量的数乘定义为
  \be
  (\alpha X_p)(f) = \alpha X_p(f)
  \ee
\end{definition}
应当注意对数乘定义的解读,$(\alpha X_p)$整体表示一个映射,它接收一个标量场$f$,作用效果是先将$f$给$X_p$作用,再乘以实数$\alpha$。因此,通过定义加法和数乘,我们把切矢量组成了一个线性空间,被叫做切空间。
\section{余切空间--切空间的对偶空间}
先补充一下什么是对偶空间。
\begin{definition}[对偶空间]
  线性空间$V$的对偶空间$V^*$是$V$上所有线性函数的集合。定义线性函数之间的加法和数乘也满足线性的关系
  \be
  (\alpha f+\beta g)(v) = \alpha f(v) + \beta g(v)
  \ee
\end{definition}
现在给线性空间$V$选择一组基底$\{e_1,\cdots,e_n \}$,这里面的$n$称为线性空间的维数,记作$n = \mathrm{dim} V$。定义一种特殊的线性函数$e^{i}$,使得
\be
e^i(e_j) = \delta^i_j
\ee
显然,可以证明,所有的线性函数都可以用$e^{i}$线性表示。所以说,$\{e^1,\cdots,e^n \}$构成对偶空间的一组基底。于是,线性空间和它的对偶空间是同维的。

现在我们定义切空间的对偶空间--余切空间。记余切空间为$T_p^*\mathcal{M}$,记余切空间中的元素为$df_p:T_p\mathcal{M}\rightarrow \mathcal{R}$,按照对偶空间的定义,$df_p$应该是切矢量的线性映射
\be
df_p(\alpha X_p + \beta Y_p ) = \alpha df_p (X) + \beta df_p(Y)
\ee
因此,找到一种线性映射就可以了,我们定义
\begin{definition}[余切矢量]
  余切矢量$df_p$的定义为
\be\label{cot:def}
df_p(X_p) \equiv X_p(f)
\ee
\end{definition}
为了证明它确实是一个对偶空间的矢量,我们证明它对作用的元素线性,并证明映射本身是线性的:首先证明对元素的线性,对于两个切矢量$X_p,\,Y_p$,有
\be
df_p(X_p+Y_p) =(X_p+Y_p)(f) = X_p(f)+Y_p(f) = df_p(X_p)+df_p(Y_p)
\ee
得证。再证明余切空间元素本身是线性的
\be
(df_p+ dg_p)(X_p) = X_p(f+g) = X_p(f)+X_p(g) = df_p(X_p) + dg_p(X_p)
\ee
可见上面的定义式~\ref{cot:def}满足对偶空间的条件。应当指出的是,我们把余切空间中的元素定义$df_p$,并不意味着它是标量场的微分,而只是一种记号,虽然我们将看到它确实是某种微分。
\section{切空间上的张量}
\subsection{切空间上张量的定义}
\begin{definition}[线性空间上的张量]
  对于一般的线性空间$V$,如果它的对偶空间为$V^*$,那么线性空间上的(s,t)型定义为一个$s+t$元多重线性函数(对每一个自变量呈线性),即
  \be
  T: \underbrace{V\times V\times \cdots \times V}_{s} \times \underbrace{V^*\times V^*\times \cdots \times V^*}_{t}
  \ee
\end{definition}
将一般的线性空间换成我们的切空间,得到切空间上张量的定义。即多重线性函数$T(X_1,\cdots,X_s; df_1,\cdots ,df_t)$。利用$\mathcal{T}_V(s,t)$表示全体$(s,t)$型张量的集合。按照这种映射的观点,$(1,0)$型张量接收一个切矢量将它映射到一个数,因此$(1,0)$型张量是一个余切矢量;$(0,1)$型张量接收一个余切矢量将它映射到一个数,因此$(0,1)$型张量是一个切矢量。
\begin{definition}[张量积]
  $V$上的$(s,t)$和$(s^{\prime},t^{\prime})$型张量$T$和$T^{\prime}$的张量积$T\otimes T^{\prime}$是一个$(s+s^{\prime},t+t^{\prime})$型张量,定义为
  \be
  \begin{aligned}
    &T\otimes T^{\prime} \left(v_1,\cdots,v_l,v_{l+1},\cdots, v_{l+l^\prime};\omega^1,\cdots, \omega^k;\omega^{k+1},\cdots,\omega^{k+k^\prime}\right) \\
    & = T\left(v_1,\cdots,v_l,v_{l+1};\omega^1,\cdots, \omega^k\right)T^{\prime}\left(v_{l+1},\cdots, v_{l+l^\prime};\omega^{k+1},\cdots,\omega^{k+k^\prime}\right)
  \end{aligned}
  \ee
\end{definition}
\begin{theorem}
  $\mathcal{T}_V(s,t)$是向量空间,它的维数为$\mathrm{dim} (s,t) = n^{s+t}$
\end{theorem}
证明略。

度规张量是切空间$T_p\mathcal{M}$上的一个$(2,0)$型对称张量,即它接收两个切矢量并返回一个实数,它是通过两个切矢量的内积定义的
\be
\langle X,Y\rangle \equiv g(X,Y)
\ee
回忆线性空间上内积的定义\footnote{注意这个内积的定义是内积满足可交换、线性、正定,并不是高中学过的坐标相乘。},我们得到度规需要满足$g(X,Y) = g(Y,X)$,这时由内积的可交换性质给出的。对于内积永远大于等于零这一点,我们在这里予以舍弃,即我们不要求切矢量的内积是正定的。物理上,这对应于相对论中的时空间隔可以为负。如果说$\forall X \in T_p\mathcal{M}$都有$g(X,X)>0$,则称具有这种度规的流形为{\color{red}黎曼流形}。

应当注意,在之前我们都是抽象地去讨论一个流形上的切矢量、余切矢量、切空间、余切空间,有了内积的定义,我们才能对流形的“形状”有某种感觉。
\subsection{度规张量给出的余切矢量}
\begin{definition}[切矢量自然对应的余切矢量]
  如果流形$\mathcal{M}$装备了度规$g$,映射$g_X(Y)\equiv g(X,Y)$,由于内积的线性性质,可以知道$g_X$是切空间到实数轴的线性映射,即余切矢量。所以称$g_X$是切矢量$X$自然对应的余切矢量。
\end{definition}
\section{用坐标来表示上面的定义}
我们知道流形$\mathcal{M}$上定义了一个开集$O$,上面定义了一个到$\mathbb{R}^n$的映射$\psi$,叫做局部坐标。
\begin{definition}[图]
  结构$(O,\psi)$称作图。一个点上所有图的集合叫做图册。
\end{definition}
两个重叠的图有坐标变换$\psi_b\circ \psi_a^{-1}: \mathbb{R}^n\rightarrow \mathbb{R}^n$,如果这个映射在流形上处处光滑,那么流形就叫做光滑的。

有了流形上点的局域坐标,我们便可以把流形上的点映射到$n$维数组上去。也就是说
\beq
p\rightarrow \left(x^1,\cdots ,x^n\right)
\eeq
\subsection{标量场的表象}
我们之前学过,标量场是流形上点到实数轴的映射,而现在流形上的点又通过局域坐标$\psi$映射到$n$维数组上,那么我们可以建立一个点的局域坐标直接到实数轴上的映射,记为$F$,故有
\beq
F\equiv f\circ \psi^{-1}
\eeq
可见$F$和$f$的联系,无非一个是把坐标映射到实数轴,而另一个是把抽象的点映射到实数轴。在今后讨论的,选好局域坐标的情况,读者可以不考虑$F,f$的区别。
\subsection{切矢量和余切矢量的分量}
回顾切矢量的定义,切矢量是标量场场到实数轴的映射。我们考虑算符
\beq
\partial_\mu (f) \equiv \frac{\partial F}{\partial x^{\mu}}
\eeq
证明上面的算符是切矢量,即证明它们满足线性和莱布尼茨律。考虑两个标量场相乘
\beq
\partial_\mu (fg) = \partial_\mu (FG) = G\partial_\mu F+F\partial_\mu G = g\partial_\mu f + f\partial_\mu g
\eeq
从而可以证明它们满足莱布尼茨律。线性的证明和上面类似。
\begin{theorem}[切矢量的基]
  所有的切矢量都可以用$\partial_\mu$唯一的线性表示,即
  \beq
  X = X^{\mu}\partial_\mu
  \eeq
\end{theorem}
从而$\partial_\mu$张成$T_p\mathcal{M}$。这个定理的证明十分困难,证明终止。有了切矢量的基,我们就可以定义余切矢量的基,回顾切矢量和余切矢量的定义
\beq
X_p(f) = df(X)
\eeq
考虑一个特殊的标量场,即$p$点的第$\nu$个坐标$x^\nu$,我们试图定义$dx^\mu$使得
\beq
\partial_\mu (x^\nu) = dx^\nu(\partial_\mu ) = \delta^{\nu}_{\mu}
\eeq
上面$dx^{\mu}$的定义是微分的正式定义,彻底解决了“无穷小”概念含糊不清的问题。因为每一个切矢量都可以用基来表示,自然地想到每一个余切矢量也可以用基来表示,我们大胆猜测有定理
\begin{theorem}[余切矢量的基]
  任意一个余切矢$df\in T_p^{*}\mathcal{M}$可以进行展开$df = f_\nu dx^{\nu}$。
\end{theorem}
证明略。考虑一个一般的余切矢量$df$,按照定义有
\beq
df(\partial_\mu) = \partial_\mu (f)
\eeq
另一方面,把这个余切矢量展开,则有
\beq
df(\partial_\mu) = (f_\nu dx^{\nu})(\partial_\mu) = f_\mu
\eeq
于是,有
\beq
f_\mu = \partial_\mu (f) = df(\partial_\mu)
\eeq
上面的式子可以这样解读:{\color{blue} 余切矢量$df$作用在切矢量的基上得到余切矢量在坐标下的表象}。

我们前面说过,切矢量是标量场到实数轴的映射,但是我们也可以不喂给它一个标量场,我们也可以喂给它一个余切矢量,定义喂给它一个余切矢量等效于喂给它一个标量场,即
\beq
X(df)\equiv X(f)
\eeq
这不禁让我们思考,余切矢量和标量场是不是一一对应的呢?这个问题留给读者去探索。如果我们引入上面的定义,我们就会得到上面类似的解读:{\color{blue} 切矢量$X$作用在余切矢量的基上得到切矢量在坐标下的分量}。
\begin{practice}
  写出上面解释对应的公式。
\end{practice}

\subsection{张量的分量}
回顾张量的定义,张量是$s$个切矢量和$t$个余切矢量到实数轴的映射,并且满足多重线性的要求,把每个切矢量和余切矢量按照基展开,有
\beq
T(X_1,\cdots, X_s; df^1,\cdots ,df^t) = X_1^{\mu_1}\cdots X_s^{\mu_s}f_{1\nu_1}\cdots f_{t\nu_t} T\left(\partial_{\mu 1},\cdots \partial_{\mu s}; dx^{\nu 1},\cdots , dx^{\nu_t}\right)
\eeq
记
\beq
T^{\nu_1\cdots \nu_t}_{\mu_1\cdots \mu_s} \equiv T\left(\partial_{\mu 1},\cdots \partial_{\mu s}; dx^{\nu 1},\cdots , dx^{\nu_t}\right)
\eeq
称为张量在给定坐标下的分量。{\color{blue} 因此,张量作用在$s$个切矢量的基和$t$个余切矢量的基上得到了张量在坐标系下的分量。}\footnote{读者可以把蓝色的字读一遍,体会这里边的关系。} 于是,原来复杂的映射就直接变成了坐标分量的简单的相乘。

在前面我们学过一种特殊的张量,它是通过内积来定义的,叫做度规张量。回顾度规张量的定义
\beq
g(X,Y) = \langle X,Y\rangle
\eeq
它的分量为
\beq
g_{\mu\nu} = g(\partial_\mu, \partial_\nu)
\eeq
于是,和上面的张量类似,这样一个抽象的映射就写成对应分量的乘积并求和
\beq
\langle X,Y\rangle = g_{\mu\nu} X^{\mu}Y^\nu
\eeq
定义一个新的余切矢量(其实在前面定义过,称为切矢量自然对应的余切矢量),它接收一个切矢量$Y$,使得
\beq
g_X(Y) = g(X,Y)
\eeq
它的坐标分量为:给这个余切矢量传进去切矢量的基
\beq
g_X(Y)_\nu = g(X,\partial_\nu) = g(X^{\mu}\partial_{\mu},\partial_\nu) = g_{\mu\nu} X^{\mu}
\eeq
在不引起歧义的情况下,定义$X_\nu = g_X(Y)_\nu$,它是我们新定义的余切矢量的坐标分量。

从上面的论述我相信你体会到了重要的一点:{\color{blue} 矢量、张量这些东西本来没有分量,就是一个抽象的映射在那里,但是因为有了坐标,它才有了分量。也就是说,矢量本来不是什么有方向的量,有了坐标之后,才能有方向}。\footnote{这种观念非常漂亮,想下去说不定搞出一种深刻的哲学。我把想象空间留给读者。}

\subsection{张量的基底}
举个例子,如果我们取两个切矢量,即两个$(0,1)$型张量,它们可以表示为
\beq
f = f^\mu \partial_\mu,\spa g = g^\mu\partial_\mu
\eeq
其中$\partial_\mu$和$\partial_\mu$,所以张量积可以表示为
\beq
T(df^1,df^2) = f^{\mu} g^{\nu} \partial_\mu(df^1) \otimes \partial_\nu(df^2)
\eeq
显然,如果$f^1,f^2$是余切空间的基底,那么就得到了张量$T$的分量。我们刚刚看到,两个$(0,1)$型张量通过张量积得到了一个$(0,2)$型张量。肯定可以证明,任何$(0,2)$型张量的基底都是$\partial_\mu \otimes \partial_\nu$,如果流形上切矢量的基底为$\partial_\mu$。可以把上面张量的基底的论述推广到任意$(s,t)$型张量。读者不妨试试。
\subsection{度规的逆、张量指标的升降}
刚刚我们定义了一个新的余切矢量$g_X(Z)$,它接收一个切矢量$Z$,现在我们定义一个新的张量$h$,它把两个刚刚的余切矢量映射到实数,并且有
\beq
h(g_X,g_Y) = g(X,Y)
\eeq
$h$就定义了切空间上的内积。
\begin{practice}
  证明上面的定义是一种内积。即证明它的可交换、线性。
\end{practice}
\begin{practice}
  证明上面的定义$h(g_X,g_Y) = g(X,Y)$等价为
  \be
  g_{\rho\sigma}h^{\sigma \nu} = \delta_{\rho}^{\spa \nu}
  \ee
  即在矩阵意义下,$h$的坐标分量矩阵是$g$坐标分量矩阵的逆矩阵。所以通常逆矩阵的分量$h^{\mu\nu}$可以记作$g^{\mu\nu}$,而不引起歧义。
  \end{practice}
我们简单的回顾一下:切矢量接收一个余切矢量得到一个实数,它的基底是下标,而它的坐标分量是上标;余切矢量接收一个切矢量得到一个实数,它的基底是上标,它的坐标分量是下标。$(s,t)$型张量接收$s$个切矢量和$t$个余切矢量,它的基底是$s$个余切矢量(上标)的基底和$t$个切矢量(下标)的基底,于是它的系数是$s$个下标。读者闲着没事睡觉的时候可以回顾一下这段。
\subsection{张量指标的缩并}
\begin{definition}[张量的缩并]
  对于一个$(s,t)$型张量$T\in\mathcal{T}_V(s,t)$的第$i$个下标与第$j$个上标的缩并定义为
  \beq
  C_i^j \left[T\right]=T(\cdots, \partial_\mu,\cdots ;\cdots,dx^{\mu},\cdots)\in \mathcal{T}(s-1,t-1)
  \eeq
\end{definition}
可以证明,$C_i^j T$与坐标选取无关。
\begin{practice}

  \begin{enumerate}[(1)]
  \item 试证明$C\left[v\otimes \omega\right] = \omega_\mu v^{\mu},\spa \forall v \in V, \omega \in V^*$
  \item 试证明$C_2^1\left[T\otimes v\right] = T(\circ,v),\spa \forall v\in V, T\in \mathcal{T}_V(2,0)$
  \item 接上题,试证明$g_X(Y) = C_2^1\left[g\otimes X\right](Y) = g(X,Y)$
  \end{enumerate}
\end{practice}
有了上面的定义,也可以定义张量指标的升降。举个例子,考虑两个张量$T = T_{\mu\nu}^\sigma dx^{\mu}\otimes dx^\nu \otimes \partial_\sigma$和$S = S_{\mu\nu}^\sigma dx^{\mu}\otimes dx^\nu \otimes \partial_\sigma$,定义一个新的$T^\prime$使得
\beq
C_1^1\left[g\otimes T\right]\otimes C_2^1\left[g\otimes S\right]=C^1_3\left[T^{\prime}\otimes S\right]
\eeq
原来的$T$是一个$(2,1)$型张量,通过这样的定义,$T$变成了一个$(3,0)$型张量。
\begin{practice}
  通过给上面的定义投食所需要的基底,证明著名的指标升降公式
  \beq
  T_{\mu\nu\rho} = g_{\rho\sigma} T^\sigma_{\mu\nu}
  \eeq
\end{practice}

\section{场}
\subsection{矢量场的定义}
我们之前在流形上每一点都定义了矢量和张量。我们知道矢量就是标量到实数轴的映射,余切矢量是切矢量到实数轴的映射。而矢量还可以理解为余切矢量到实数轴的映射。我们之前说过标量场就是流形上每一个点到实数轴的映射。现在我们试图定义矢量场的概念,我们直接将上面的“标量、矢量”等词汇替换成“标量场、矢量场”。那么词汇“实数轴”应该怎么替换呢?

数学上,我们作如下定义
\begin{definition}[矢量场]
  流形$\mathcal{M}$上的矢量场$X$定义为光滑标量场之间的映射
  \beq
  X: f\rightarrow X(f) ,\spa f,X(f)\in C^\infty
  \eeq
\end{definition}
上面的定义实际上很好理解,我们知道流形上一个矢量实际上是标量场到实数轴上的映射,也就是说,一个矢量是一个映射,吃掉一个标量,吐出一个实数。那么如果矢量到处吃标量,把每一点的标量都吃了,那么它在每一点都吐出一个实数,这些实数又构成了一个标量场。所以说矢量场就是标量场到标量场的映射。

\subsection{矢量场对易子与李导数}
考虑两个矢量的复合,即将一个标量场通过矢量场$X$映射到另一个标量场,再将这个新产生的标量场通过另一个矢量场$Y$映射产生又一个新的标量场$(Y\circ X)(f)$。还记得如果一个映射是矢量场,它还需要满足线性和莱布尼茨律。考虑
\beq
\begin{aligned}
  (X\circ Y)(fg) &= X(Y(fg)) = X(fY(g)+gY(f))\\
  &= X(f)Y(g) + fX(Y(g))+X(g)Y(f) + gX(Y(f))\\
  &\ne fX(Y(g))+gX(Y(f))
\end{aligned}
\eeq
将$X,Y$的位置调换并相减,可以证明$[X,Y] = XY-YX$是矢量场。可以验证,对易子的坐标表象为
\beq
    [X,Y] = (X^{\mu}\partial_{\mu}Y^\nu - Y^\mu \partial_\mu X^\nu)\partial_\nu
\eeq

如果我们固定$X$变化$Y$,就得到了一个矢量场到矢量场的映射
\beq
\mathcal{L}_X(Y) = [X,Y]
\eeq
$\mathcal{L}_X$又被称作李导数。

同理,我们还可以定义余切矢量场和标量场。我累了,定义终止。
\subsection{局部坐标变换}
我们之前通过在流形上的开子集引入坐标$x^\mu$,从而引入了切空间的基$\partial_\mu$,现在在同一地点,考虑另一个开子集而引入的另一组坐标$u^\nu$,那么它也会定义切空间的一组基$\partial /\partial u^{\nu}$,这两组坐标之间如何变换?

回顾一下切空间上的切矢量是将一个标量映射到实数轴上,而标量在引入了坐标之后又可以表示成多元函数,因此这里切空间上切向量的坐标变换公式就和多元函数的坐标变换公式相同,即
\beq
\frac{\partial }{\partial x^{\mu}} = \frac{\partial u^\nu}{\partial x^{\mu}}\frac{\partial}{\partial u^\nu}
\eeq
因为切空间上任何一个切矢量都可以表示成基底的线性组合,因此,对于同一个切矢量$v$,它在坐标$x$下的分量$v(x)^\mu$和在坐标$u$下的表象$v(u)^\nu$的关系为
\beq
v = v(x)^\mu \frac{\partial}{\partial x^\mu} = v(u)^\nu \frac{\partial}{\partial u^\nu}
\eeq
进而利用上面的关系,将等式右边的基换成等式左边的基,从而得到
\beq
v(x)^\mu = v(u)^\nu \frac{\partial x^\mu}{\partial u^{\nu}}
\eeq

对于余切矢量,在另一组坐标$u^\nu$下定义了一组余切空间的基$du^\nu$,它们满足
\beq
du^{\nu}\left(\frac{\partial}{\partial u^\mu}\right) = \delta^\nu_\mu = dx^{\nu}\left(\frac{\partial}{\partial x^\mu}\right)
\eeq
再利用切矢量基底的变换公式,就得到了余切矢量基底的变换公式
\beq
dx^\mu = \frac{\partial x^{\mu}}{\partial u^{\nu}}du^\nu
\eeq
进而将任何一个余切矢量使用基底展开,就能得到余切矢量的分量坐标变换公式。
\section{抽象指标记号}
可以使用两种方法表示张量,第一种方法当然是使用映射的观点,这种观点论述起来非常麻烦;而第二种方法是直接使用张量的分量来代表张量。在这里我们要注意,之前如果我们不讨论张量的指标升降,我们可以简单的把张量的分量记为$T^{j_1\cdots j_S}_{i_1\cdots i_t}$而不引起歧义。但是如果涉及指标的升降,我们最好将上下标错开,以表明张量对应的映射次序,例如一个$(1,2)$型张量记为$T_c^{\spa ab}$。我们接下来定义一些记号
\begin{definition}[对称张量]
  $T\in \mathcal{T}_V(2,0)$是对称的,如果$T(u,v) = T(v,u)$,$\forall u,v \in V$。
\end{definition}
\begin{definition}
  $(2,0)$型张量$T_{ab}$的对称部分记作$T_{(ab)}$,反称部分记作$T_{[ab]}$,定义为
  \be
  T_{(ab)} = \frac{1}{2}\left(T_{ab}+T_{(ba)}\right)
  \ee
\end{definition}
一般地,$(l,0)$型张量$T_{a_1\cdots a_l}$的对称部分和反称部分的定义为
\bea
T_{(a_1\cdots a_l)} = \frac{1}{l!}\sum_\pi T_{a\pi(1)\cdots a_{\pi(l)}},
T_{[a_1\cdots a_l]} = \frac{1}{l!}\sum_\pi \epsilon_\pi T_{a\pi(1)\cdots a_{\pi(l)}}
\eea
\section{习题1}
\begin{problem}
  使用张量变换规律,求出三维欧式度规在求坐标系中全部分量$g_{\mu\nu}^\prime$。
\end{problem}
\begin{problem}
  证明如果$T_{abcd} = T_{a[bc]d} = T_{ab[cd]}$,那么$T_{a[bcd]}$。
\end{problem}
\begin{problem}
  假设有一个$(0,2)$型张量的分量为$V^{\mu\nu}$,考虑两个$(2,0)$型对称和反对称张量的分量$S_{\mu\nu},A_{\mu\nu}$,证明$V^{\mu\nu}S_{\mu\nu} = V^{(\mu\nu)}S_{\mu\nu}$。
\end{problem}



\end{document}
